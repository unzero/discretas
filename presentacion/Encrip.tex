\documentclass{beamer}

\usepackage[spanish]{babel}
\usepackage[latin1]{inputenc}
\usepackage[dvips]{graphics} % LaTeX
\usepackage{graphicx}
\DeclareGraphicsExtensions{.bmp, .png, .jpg, .gif}

\usetheme{Warsaw}
\usecolortheme{crane}
\useoutertheme{shadow}

\title
{
  \textbf{Proyecto Final de Matem\'aticas Discretas \\ 
  Fractales Aplicados a la criptograf\'ia \\ 
  Grupo: K-ON}
}

%DATOS SOBRE LOS AUTORES
\author
{
  \\ \\ \\ \\ \\ \\ \\ \\ \\ \\ \\ \\ 
  \textbf{David Ricardo Dager} \\ 
  \textbf{Manuel Arturo Ram\'irez} \\ 
  \textbf{Ciro Iv\'an Garc\'ia} \\ \\ \\
  \textbf{Presentado a: Andres Villaveces Ni\~no} \\ \\ \\ \\ \\ \\ \\ \\ \\ \\ \\ \\ \\
  \textbf{Universidad Nacional de Colombia} \\
  \textbf{Facultad de Ingenier\'ia} \\ 
  \textbf{Departamento de Ingenier\'ia de Sistemas e Industrial}
}

\newcommand{\IncludeImage}[4]{
  \begin{figure}[ht!]
    \centering
      \includegraphics[width=#2mm]{#1.png}
      %\caption{#3}
      %\label{#4}
  \end{figure}
}

\begin{document}
\begin{frame}
 \titlepage
\end{frame}
\begin{frame}
 \frametitle{Fractales}
 ``Un fractal es la representaci\'on gr\'afica del caos'' (Gutierrez y Hott).\\
 Caracter\'isticas a resaltar:
 \begin{itemize}
  \item Comportamiento ca\'otico.
  \item Autosimilares.
  \item Representaci\'on algor\'itmica ``simple''.
 \end{itemize}

\end{frame}

\begin{frame}
 \frametitle{Caos (matem\'atico)}
 \begin{itemize}
  \item An\'alisis no convencional.
  \item Ligado fuertemente a condiciones iniciales.
  \item Efecto mariposa es una poripiedad intr\'inseca del caos.
  \item Trayectoras cuasi peri\'odicas.
  \item Atractores extra\~nos.
 \end{itemize}

\end{frame}

\begin{frame}
 \frametitle{Collatz}
 Conjetura de Collatz :
 \IncludeImage{./images/cf}{95}{Conjetura de Collatz}{a}
\end{frame}

\begin{frame}
 \frametitle{Transformaci\'on de Baker}
 Versi\'on general:
 Sea N en conjunto N=\{0,1,2..,n-1\} de n\'umeros enteros y sea $\lambda$ un conjunto de enteros, $\lambda$=$\{\lambda_{1}, \lambda_{2},...,\lambda_{k}\}$, que satisface las siguientes propiedades:

\begin{itemize}
	\item $\lambda_{1}+\lambda_{2}+...+\lambda_{k}$=n.
	\item $\lambda_{i}\mid n$ $\forall$ i $\epsilon$ \{1,2,...,k\}
\end{itemize}

Definimos la transformaci\'on discreta de Baker $T_{N,\lambda}$ : N $\times$ N $\longmapsto$ N $\times$ N de la siguiente manera:

\begin{equation}
  T_{N,\lambda}(x,y) = [q_{i}(x-\sigma_{i}) + y \ mod \ q_{i} , \frac{1}{q_{i}}(y - y \ mod \ q_{s} ) + \sigma_{i}
\end{equation}

De donde $\sigma_{1}$ := 0 y $\sigma_{i}$ := $\lambda_{1}+...+\lambda_{i-1}$ para $2 \leq i < k$, $q_{i}$ := $\frac{n}{\lambda_{i}}$ y (x,y) $\epsilon$ [$\sigma_{i},\sigma_{i}+\lambda_{i})\times N $.
 
\end{frame}

\begin{frame}
\frametitle{Transformaci\'on de Baker}
Tomado de Jiri:
 \IncludeImage{./images/j}{70}{imagen tomada de Jiri}{b}
\end{frame}

\begin{frame}
\frametitle{ Transformaci\'on de Baker II}
  Sea N en conjunto N=\{0,1,2..,n-1\} de n\'umeros enteros y sea $\lambda$ un conjunto de enteros, $\lambda$=$\{\lambda_{1}, \lambda_{2},...,\lambda_{k}\}$, que satisface las siguientes propiedades:

\begin{itemize}
	\item $\lambda_{1}+\lambda_{2}+...+\lambda_{k}$=n.
	\item $\lambda_{i}\mid n$ $\forall$ i $\epsilon$ \{1,2,...,k\}
\end{itemize}

Definimos la transformaci\'on discreta de Baker $B_{N,\lambda}$ : N $\times$ N $\longmapsto$ N $\times$ N de la siguiente manera:

\begin{equation}
  B_{N,\lambda}(x,y) = ( n - \sigma_{i} - \lfloor\frac{n-(x+1)}{q_{i}}\rfloor -1 , \frac{y-\sigma_{i}}{q_{i}} + [n-(x+1)] \ mod \ q_{i} ) 
\end{equation}

De donde $\sigma_{1}$ := 0 y $\sigma_{i}$ := $\lambda_{1}+...+\lambda_{i-1}$ para $1 \leq i < k$, $q_{i}$ := $\frac{n}{\lambda_{i}}$ y (x,y) $\epsilon$ $ N \times [\sigma_{i},\sigma_{i}+\lambda_{i})$.

\end{frame}

\begin{frame}
 Gracias.
\end{frame}


\end{document}
