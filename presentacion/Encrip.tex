\documentclass[dvipdfm]{beamer}
\usepackage{eso-pic}
\usepackage{graphicx} 
\usepackage{anysize}
\usepackage[spanish]{babel}
\usepackage{amsmath}

\usetheme{Warsaw}
\usecolortheme{crane}
\useoutertheme{shadow}

\title
{
  \textbf{Proyecto Final de Matem\'aticas Discretas \\ 
  Fractales Aplicados a la criptograf\'ia \\ 
  Grupo: K-ON}
}

%DATOS SOBRE LOS AUTORES
\author
{
  \\ \\ \\ \\ \\ \\ \\ \\ \\ \\ \\ \\ 
  \textbf{David Ricardo Dager} \\ 
  \textbf{Manuel Arturo Ram\'irez} \\ 
  \textbf{Ciro Iv\'an Garc\'ia} \\ \\ \\
  \textbf{Presentado a: Andres Villaveces Ni\~no} \\ \\ \\ \\ \\ \\ \\ \\ \\ \\ \\ \\ \\
  \textbf{Universidad Nacional de Colombia} \\
  \textbf{Facultad de Ingenier\'ia} \\ 
  \textbf{Departamento de Ingenier\'ia de Sistemas e Industrial}
}

\newcommand{\IncludeImage}[4]{
  \begin{figure}[ht!]
    \centering
      \includegraphics[width=#2mm]{#1.png}
      %\caption{#3}
      %\label{#4}
  \end{figure}
}

\begin{document}

\begin{frame}
 \titlepage
\end{frame}

\begin{frame}
 \frametitle{Fractales}
 ``Un fractal es la representaci\'on gr\'afica del caos'' (Gutierrez y Hott).\newline
 Caracter\'isticas a resaltar:
 \begin{itemize}
  \item Comportamiento ca\'otico.
  \item Autosimilares.
  \item Representaci\'on algor\'itmica ``simple''.
 \end{itemize}

\end{frame}

\begin{frame}
 \frametitle{Caos (matem\'atico)}
 \begin{itemize}
  \item An\'alisis no convencional.
  \item Ligado fuertemente a condiciones iniciales.
  \item Efecto mariposa es una poripiedad intr\'inseca del caos.
  \item Trayectoras cuasi peri\'odicas.
  \item Atractores extra\~nos.
 \end{itemize}

\end{frame}

\begin{frame}
 \frametitle{Collatz}
 Conjetura de Collatz :
 \IncludeImage{./images/cf}{95}{Conjetura de Collatz}{a}
\end{frame}

\begin{frame}
 \frametitle{Transformaci\'on de Baker}
 Versi\'on general:
 Sea U el cuadrado unitario se define la transformaci\'on de Baker de la siguiente manera:

\begin{equation}
T(x,y) = (\frac{1}{p_{i}}(x-F_{i}) , p_{i}y+F_{i})
\end{equation}

De donde (x,y) $\epsilon$ [$F_{i},F_{i}+p_{i})\times$ [0,1).
 
\end{frame}

\begin{frame}
\frametitle{Transformaci\'on de Baker}
Tomado de Jiri:
 \IncludeImage{./images/j}{70}{imagen tomada de Jiri}{b}
\end{frame}

\begin{frame}
\frametitle{ Transformaci\'on de Baker II}
  Sea N en conjunto N=\{0,1,2..,n-1\} de n\'umeros enteros y sea $\lambda$ un conjunto de enteros, $\lambda$=$\{\lambda_{1}, \lambda_{2},...,\lambda_{k}\}$, que satisface las siguientes propiedades:

\begin{itemize}
	\item $\lambda_{1}+\lambda_{2}+...+\lambda_{k}$=n.
	\item $\lambda_{i}\mid n$ $\forall$ i $\epsilon$ \{1,2,...,k\}
\end{itemize}

Definimos la transformaci\'on discreta de Baker $B_{N,\lambda}$ : N $\times$ N $\longmapsto$ N $\times$ N de la siguiente manera:

\begin{equation}
  B_{N,\lambda}(x,y) = ( n - \sigma_{i} - \lfloor\frac{n-(x+1)}{q_{i}}\rfloor -1 , \frac{y-\sigma_{i}}{q_{i}} + [n-(x+1)] \ mod \ q_{i} ) 
\end{equation}

De donde $\sigma_{1}$ := 0 y $\sigma_{i}$ := $\lambda_{1}+...+\lambda_{i-1}$ para $1 \leq i < k$, $q_{i}$ := $\frac{n}{\lambda_{i}}$ y (x,y) $\epsilon$ $ N \times [\sigma_{i},\sigma_{i}+\lambda_{i})$.

\end{frame}

\begin{frame}
 \frametitle{Cifrar el mensaje}
 El cifrado del mensaje se hace caracter a caracter:
 \begin{itemize}
 \item Tomar el caracter de la cadena, $\alpha_{i}$.
 \item Elegir un n\'umero, entero, aleatorio dentro del dominio [a,b], $\varphi_{i}$.
 \item Evaluar $f_{encp}(\alpha_{i},\varphi_{i})$ obteniendo $\mu_{i}$.
 \item Almacenar $\varphi_{i}$,$\alpha_{i}$,$\mu_{i}$.
 \item Si a\'un quedan caracteres volver a 1, si no transmitir.

 \end{itemize}

\end{frame}

\begin{frame}
 \frametitle{$f_{encryp}$}
 La funci\'on de encriptaci\'on (basada en la conjetura de Collatz) es la siguiente:
 
 \[ x_{1} = 
\begin{cases} 
	\frac{x}{2} \text { mod  9} & \text {si } x \text { es par} \\ 
	(3x+1) \text { mod 9} & \text {si  x} \text { es impar} 
\end{cases} 
\] 
Por la naturaleza de la conjetura se realiza un casteo para los valores 3, 7 y 8.
 
\end{frame}


\begin{frame}
\frametitle{Descifrar el mensaje}
Para descifrar el mensaje debemos conocer la matriz de cifrado, ubicaremos el caracter cifrado $\mu$ en la matriz con la siguiente funci\'on

\[ \mu_{decrypted} = 
\begin{cases} 
	0 	& \text { si } \mu = 0 \\ 
	2*\mu 	& \text { si } 0 < \mu < 4 \\
	1 	& \text { si } \mu = 4 \\
	3 	& \text { si } \mu = 5 \\
	8	& \text { si } \mu = 6 \\
	5 	& \text { si } \mu = 7 \\
	7	& \text { si } \mu = 8 \\
\end{cases} 
\] 
El proceso se realiza $\varphi$ veces, que es la cantidad de iteraciones usadas para cifrar a $\mu$. 

\end{frame}


\begin{frame}
 Gracias.
\end{frame}


\end{document}
