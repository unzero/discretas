\section{Algoritmo para el descifrado}
Para descifrar la cadena usamos un algortimo 'simple', c\'omo mencion\'abamos anteriormente. Son necesarios dos caracteres; $\mu$ y $\varphi$, que representan al caracter cifrado y al n\'umero de iteraciones respectivamente.

Empezaremos con un contador $cont = 1$, el cu\'al indicar\'a si se encuentra en una posici\'on par o impar, este hecho modifica la lectura de $\mu$, dada de la siguiente manera:

\begin{center}
$\mu$, $\varphi$ Si \emph{cont} es par 
\end{center}
\begin{center}
$\varphi$, $\mu$ Si \emph{cont} es impar
\end{center}

Se debe tener en cuenta que $cont$ se aumenta en una unidad cada dos caracteres. Ubicamos a $\varphi$ en nuestra matriz $\Psi$ de cifrado y desciframos su valor num\'erico por medio de la ecuaci\'on:

\begin{equation}
 pos = (x * n) + y
\end{equation}

Siendo $n$ la dimensi\'on de nuestra matriz, en este caso $n = 9$. Como ya conocemos el valor num\'erico de $\varphi$ procedemos a descifrar $\mu$ el n\'umero de iteraciones correspondientes, para esto usamos la ecuaci\'on:

\[ \mu_{decrypted} = 
\begin{cases} 
	0 	& \text { si } \mu = 0 \\
	2*\mu 	& \text { si }  0 < \mu < 4\\
	3 	& \text { si } \mu = 5 \\
	8 	& \text { si } \mu = 6 \\
	\frac{\mu+8}{3} & \text { si } \mu = 7 \\
	6 	& \text { si } \mu = 8 \\
\end{cases} 
\] 

*Se tienen en cuenta los tres casos especiales 5, 6 y 8 que fueron casteados.

Dicha operaci\'on se realiza $\varphi$ veces con cada par de coordenadas ($x$, $y$). Al final se proceder\'a con transmitir el mensaje obtenido el cu\'al debe ser igual al original.
