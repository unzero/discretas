\section{Introducci\'on}
Gracias a la expansi\'on de la red y de la informaci\'on en estos ultimos a\~nos; el problema de transmitir informaci\'on de forma correcta, eficiente, r\'apida y segura es uno de los mayores retos para el mundo entero. Es aqu\'i donde la criptograf\'ia juega un papel importante en la vida cotidiana de todo ser humano, permitiendole comunicarse a plenitud con el entorno que le rodea sin temor alguno. 

Hasta la fecha existen distintos modelos de criptograf\'ia tales como RSA, cu\'antica, asim\'etrica, etc. Los cuales cuentan con un desarrollo e investigaci\'on muy avanzado. Sin embargo poco a poco han emergido distintos modelos que buscan resolver el problema original desde otras perspectivas, indagando en \'areas de la matem\'atica que a\'un no han sido exploradas; como es el caso de la criptograf\'ia basada en fractales, modelo en el cual nos preguntamos por la naturaleza de la geometr\'ia fractal y del caos, en busca de soluciones m\'as sencillas sin dejar de lado los aspectos importantes de un buen cifrado.

Tal como como es expuesto por Nadia y Mohamad \cite{Nadia} los sistemas de cifrado basados en fractales\footnote{De ahora en adelante nos referiremos a ellos como SCF.} son sistemas cuyo comportamiento no es f\'acil determinar o predecir, siendo este un factor decisorio a la hora de realizar criptoan\'alisis, ya que muchas de las t\'ecnicas usadas en otros modelos tradicionales de cifrado no se ajustan a la realidad de estos sistemas.

A su vez Ljupco \cite{Ljupco} en su trabajo explora los alcances y limitaciones que tienen los SCF, algunas de las cuales llegan a tener un nivel de complejidad tan alto que no ser\'an abordados en el presente trabajo. Estas limitaciones son a su vez un factor motivante en el desarrollo del presente trabajo; es por ello que hemos definido \textbf{\emph{INSIGNIAS}} con el fin de resaltar los aspectos m\'as importantes para el trabajo.

Comenzaremos explorando por qu\'e resulta conveniente para nosotros el uso de fractales, Secci\'on 2. La secci\'on 3 describe el algoritmo de cifrado a un nivel de detalle alto. La matriz de cifrado junto con la funci\'on de cifrado, coraz\'on del presente trabajo, son trabajadas en la secci\'on 4. El resultado y los detalles de transmici\'on para un mensaje ser\'an expuestos en la secci\'on 5. La secci\'on 6 se encargara del algoritmo para decifrar el mensaje y su funcionamiento. Por \'ultimo presentamos las conclusiones del trabajo en la secci\'on 7.

