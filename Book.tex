%CODIGO FUENTE TRABAJO ESCRITO : ESTILO LIBRO
%PROYECTO FINAL DE MATEMATICAS DISCRETAS
%UNIVERSIDAD NACIONAL DE COLOMBIA
%JUAN CAMILO NEIVA
%DAVID RICARDO DAGER MORA
%MANUEL ARTURO RAMIREZ PIZCO
%CIRO IVAN GARCIA LOPEZ
%PRESENTADO A: ANDRES VILLAVECES NIÑO

%TIPO DE DOCUMENTO, PAQUETES ADICIONALES
\documentclass[a4paper,10pt]{article}
\usepackage{eso-pic}
\usepackage{graphicx} 
\usepackage{anysize}
\usepackage[spanish]{babel}
\usepackage{amsmath}
\usepackage{graphicx}
\usepackage{tikz}
%SIRVE PARA GENERAR AUTOMATICAMENTE LAS ENUMERACIONES EN DEFINICIONES
\newtheorem{defi}{\it DEFINICI\'ON}
\newtheorem{algo}{\it ALGORITMO}
\newtheorem{insig}{\it INSIGNIA}
%TITULO DEL DOCUMENTO
\title
{
  \textbf{Proyecto Final de Matem\'aticas Discretas \\ 
  Fractales Aplicados a la criptograf\'ia \\ 
  Grupo: K-ON}
}

%DATOS SOBRE LOS AUTORES
\author
{
  \\ \\ \\ \\ \\ \\ \\ \\ \\ \\ \\ \\ 
  \textbf{David Ricardo Dager} \\ 
  \textbf{Manuel Arturo Ram\'irez} \\ 
  \textbf{Ciro Iv\'an Garc\'ia} \\ \\ \\
  \textbf{Presentado a: Andres Villaveces Ni\~no} \\ \\ \\ \\ \\ \\ \\ \\ \\ \\ \\ \\ \\
  \textbf{Universidad Nacional de Colombia} \\
  \textbf{Facultad de Ingenier\'i�a} \\ 
  \textbf{Departamento de Ingenier\'i�a de Sistemas e Industrial}
}

%MODIFICADORES DEL DOCUMENTO
%NO GENERA FECHA
%\date{}

%MARGENES
\marginsize{2.5cm}{2.5cm}{2.5cm}{2.5cm}



%RUTINA PARA PONER LA IMAGEN DE FONDO DE LA PRIMERA PAGINA 
\newcommand{\BackgroundImage}[1]{
  \AddToShipoutPicture*{
    \put(0,0){
      \parbox[b][\paperheight]{\paperwidth}{
	\vfill
	\centering
	\includegraphics[width=15cm,height=15cm,
	keepaspectratio]{#1.png}
	\vfill
      }
    }
  }
}

%RUTINA PARA INSERTAR IMAGENES AL TRABAJO
%RECIBE
%RUTA DE LA IMAGEN
%NOMBRE DE LA IMAGEN
%REFERENCIA DE LA IMAGEN <-- PARA PODER NOMBRARLA 
\newcommand{\IncludeImage}[3]{
  \begin{figure}[ht!]
    \centering
      \includegraphics[width=#2mm]{#1.png}
      \caption{#3}
      %\label{#4}
  \end{figure}
}


\begin{document}
  %PONE LA IMAGEN DE FONDO
  \BackgroundImage{escudo_un}    
  \maketitle
  \newpage

  %GENERA LA TABLA DE CONTENIDOS
  \tableofcontents
  %\newpage

  %modulos del proyecto
  \section{Introducci\'on}
Gracias a la expansi\'on de la red y de la informaci\'on en estos ultimos a\~nos; el problema de transmitir informaci\'on de forma correcta, eficiente, r\'apida y segura es uno de los mayores retos para el mundo entero. Es aqu\'i donde la criptograf\'ia juega un papel importante en la vida cotidiana de todo ser humano, permitiendole comunicarse a plenitud con el entorno que le rodea sin temor alguno. 

Hasta la fecha existen distintos modelos de criptograf\'ia tales como RSA, cu\'antica, asim\'etrica, etc. Los cuales cuentan con un desarrollo e investigaci\'on muy avanzado. Sin embargo poco a poco han emergido distintos modelos que buscan resolver el problema original desde otras perspectivas, indagando en \'areas de la matem\'atica que a\'un no han sido exploradas; como es el caso de la criptograf\'ia basada en fractales, modelo en el cual nos preguntamos por la naturaleza de la geometr\'ia fractal y del caos, en busca de soluciones m\'as sencillas sin dejar de lado los aspectos importantes de un buen cifrado.

Tal como como es expuesto por Nadia y Mohamad \cite{Nadia} los sistemas de cifrado basados en fractales\footnote{De ahora en adelante nos referiremos a ellos como SCF.} son sistemas cuyo comportamiento no es f\'acil determinar o predecir, siendo este un factor decisorio a la hora de realizar criptoan\'alisis, ya que muchas de las t\'ecnicas usadas en otros modelos tradicionales de cifrado no se ajustan a la realidad de estos sistemas.

A su vez Ljupco \cite{Ljupco} en su trabajo explora los alcances y limitaciones que tienen los SCF, algunas de las cuales llegan a tener un nivel de complejidad tan alto que no ser\'an abordados en el presente trabajo. Estas limitaciones son a su vez un factor motivante en el desarrollo del presente trabajo; es por ello que hemos definido \textbf{\emph{INSIGNIAS}} con el fin de resaltar los aspectos m\'as importantes para el trabajo.

Comenzaremos explorando por qu\'e resulta conveniente para nosotros el uso de fractales, Secci\'on 2. La secci\'on 3 describe el algoritmo de cifrado a un nivel de detalle alto. La matriz de cifrado junto con la funci\'on de cifrado, coraz\'on del presente trabajo, son trabajadas en la secci\'on 4. El resultado y los detalles de transmici\'on para un mensaje ser\'an expuestos en la secci\'on 5. La secci\'on 6 se encargara del algoritmo para decifrar el mensaje y su funcionamiento. Por \'ultimo presentamos las conclusiones del trabajo en la secci\'on 7.


  \section{Fractales}

Los fractales son representaciones geom\'etricas de gran inter\'es para cualquier persona ya sea por la belleza de sus formas, su simplicidad o por la naturaleza que los gobierna. M\'as all\'a de estas razones existe un mundo matem\'atico que poco a poco ha venido emergiendo y ha sido aplicado en campos como compresi\'on de archivos, generaci\'on de graficas, simulaciones y por su puesto en la criptograf\'ia; campo en el cual nos puede llevar a encontrar nuevos resultados de gran inter\'es. 

Hasta el momento hemos hablado sobre fractales sin embargo no se ha dado una definici\'on rigurosa. Rubiano define un fractal de la siguiente manera ''Un fractal es un subconjunto del plano que es autosimilar y cuya dimensi\'on fractal excede a su dimensi\'on topol\'ogica" \cite{Rubiano} sin embargo, ¿Qu\'e es dimensi\'on fractal? ¿Qu\'e es dimensi\'on topol\'ogica?. Para poder definirlas se deben conocer conceptos como cubiertas y nociones de subconjuntos llamados abiertos que no entraremos a definir debido a su complejidad y los l\'imites del trabajo.

Es tambi\'en importante resaltar lo que nos dicen Gutierrez y Hott \cite{Gutierrez} sobre la relaci\'on del caos y los fractales, "Los fractales son la representaci\'on gr\'afica del caos".

Los fractales presentan propiedades algunas de las cuales son de mayor inter\'es para el presente trabajo, algunas de ellas son:
 
\begin{enumerate}
  \item Tienen un comportamiento ca\'otico. 
  \item Poseen una dimensi\'on fraccionaria, infinita.
  \item Son autosimilares
  \item Son representados por un algoritmo \emph{simple}\footnote{En el contexto del presente trabajo simple hace referencia a la facilidad del algoritmo conociendo los valores relevantes y la ecuaci\'on correspondiente para descifrado.}
\end{enumerate}

El comportamiento ca\'otico de los fractales hace de los SCF sistemas eficientes y seguros para la generaci\'on de un mensaje cifrado, tal cual como es sugerido por Nadia y Mohamad \cite{Nadia,Nadia2} gracias al caos el trabajo que se debe realizar para vulnerar un SCF se vuelve costoso, necesita un an\'alisis no convencional y requiere de una gran cantidad de tiempo, tanto humano como de m\'aquina. El llamado \emph{Efecto mariposa} propiedad intrinseca del caos y por ende de los fractales actua directamente sobre el sistema reflejandose en \emph{dado un peque\~no cambio sobre el mensaje original, la cadena cifrada sufrir\'a un cambio brusco}.

Por otra parte la propiedad de ser autosimilar es explotado en el dise\~no del algoritmo que represente al fractal. En una primera aproximaci\'on este algoritmo puede estar dado en t\'erminos de una funci\'on recursiva, infinita, que debe ser acotada en un momento dado ya sea por los limites f\'isicos de la m\'aquina o por alg\'un otro l\'imite impuesto por el usuario. 

Se han mencionado algunas de las ventajas que poseen los fractales para el SCF sin embargo Howell y Reese \cite{Howell} en su trabajo exponen el mayor de los problemas en el uso de fractales aplicados a la criptograf\'ia; la complejidad para decifrar la informaci\'on, lo cual es un paso obligatorio para la validaci\'on del sistema. Durante el desarrollo se ha prestado atenci\'on a este problema e intentaremos dar un tratamiento a la soluci\'on propuesta de tal forma que obtengamos un sistema verificable.



  \section{Desarrollo del problema}

Luego de explorar diversas fuentes en busca de informaci\'on sobre cifrado basado en fractales y caos se ha llegado a la conclusi\'on de reunir varios aspectos de ellos con el fin de generar un SCF de alta calidad. Entre las ideas m\'as relevantes cabe resaltar las propuestas de Howell y Reese \cite{Howell}, Nadia y Muhammad \cite{Nadia2}, Makris y Ioannis \cite{Makris}, Pichler y Scharinger \cite{Pichler}, Jiri \cite{Jiri} y por \'ultimo la propuesta de Yuen y Wong \cite{Yuen}.

Con el fin de poder tratar los detalles del SCF es prudente hablar de las definiciones base o preliminares.

\begin{defi}:
Definimos como $\Omega$ el conjunto de car\'acteres \{a,b,...,z,A,B,...,Z,0,1,2...,9,+,-,*,/,$<$,$>\\$,\#,$\gamma$,?,\%,\$,\{,\},coma,punto,[,],(,)\}; donde $\gamma$ es el espacio en blanco y sea $\Omega^{*}$ el conjunto de todas las cadenas sobre $\Omega$.
\end{defi}

\begin{defi}:
Un SCF ($\xi$) es una **funci\'on para la cual
$$\xi:\Omega^{*} \longmapsto \Omega^{*}$$
$$    x \longmapsto x^{c} $$
De donde x $\epsilon$ $\Omega^{*}$ y $x^{c}$ se dice la cadena cifrada de \emph{x}.
\end{defi}

La definici\'on 1 nos permite plantear nuestra primera insignia.\footnote{Las insignias son observaciones de gran importancia para el desarrollo del trabajo}

\begin{insig} : 
El SCF presentado est\'a limitado a trabajar unicamente con conjunto de car\'acteres $\Omega$.
\end{insig}

El alfabeto de entrada es considerado una insignia dado que indica la directriz que deben seguir los textos de entrada para el SCF; para un car\'acter no pertenesca a este conjunto el comportamiento del SCF no ha sido determinado. Con nuestras las dos primeras definiciones podemos describir el funcionamiento interno del SCF, para este fin se ha escrito el siguiente algoritmo.

\begin{algo}
SCF ($\xi$) : Dada una cadena x $\epsilon$ $\Omega^{*}$ con \textbf{j} car\'acteres; se procesar\'a de la siguiente manera.
\begin{enumerate}
 \item {Generar la matriz de cifrado.}
 \item { Cifrar el mensaje
	\begin{enumerate}
	  \item Tomar el siguiente car\'acter de la cadena, $\alpha_{i}$.
	  \item Elegir un n\'umero, entero, aleatorio dentro del dominio [a,b], $\varphi_{i}$.
	  \item Evaluar $f_{encp}(\alpha_{i},\varphi_{i})$ obteniendo $\mu_{i}$.
	  \item Almacenar $\varphi_{i}$,$\alpha_{i}$,$\mu_{i}$.
	 \item S\'i aun quedan car\'acteres volver a 1, si no transmitir.
	\end{enumerate}
 }
\end{enumerate}
\end{algo}

Al finalizar nuestro algoritmo sobre la cadena \emph{x} tendremos la siguiente configuraci\'on para las entradas y salidas.

\begin{center}
  \begin{tabular}{ |c | c | c | }
    \hline
    Llave & Entrada & $f_{encp}$ \\	
    \hline
    $\varphi_{1}$ & $\alpha_{1}$ & $\mu_{1}$ \\
    \hline
    $\varphi_{2}$ & $\alpha_{2}$ & $\mu_{2}$ \\
    \hline
    : & : & : \\
    \hline
    $\varphi_{j}$ & $\alpha_{j}$ & $\mu_{j}$ \\
    \hline
  \end{tabular}
\end{center}

Del algoritmo 1 podemos introducir las siguientes preguntas a\'un sin responder.

\begin{itemize}
	\item ?`Qu\'e es la matriz de cifrado?
	\item ?`Qui\'enes son los extremos del dominio?
	\item ?`Qu\'e es $f_{encp}(\alpha,\varphi)$?
	\item ?`Qui\'en ser\'a $x^{c}$?
\end{itemize}

Las preguntas sobre la matriz de cifrado, los extremos del dominio y $f_{encp}(\alpha,\varphi)$ se trabajan en la secci\'on 4. Igualmente la primera esta relacionada con la forma en la cual debemos transmitir el mensaje, motivo por el cual ser\'a resuelta en la secci\'on 5.

  \section{Matriz de cifrado y funci\'on de encriptaci\'on $f_{encryp}$}

En la secci\'on 4 se menciono la matriz de cifrado y la funci\'on $f_{encryp}$; en la presente secci\'on buscamos presentar al m\'aximo de detalles estos dos componentes del trabajo, empezaremos hablando de la matriz de cifrado y su papel dentro del trabajo.

\begin{defi} Matriz de Cifrado($\Psi$): 
	Es una matriz de tama\~no 8$\times$8 la cual almacena todos los car\'acteres de $\Omega$.
\end{defi}

\begin{defi} Transformaci\'on discreta de Baker\footnote{Algunos autores tambi\'en se refieren como Permutaciones de Bernoulli.}.

Sea N en conjunto N=\{0,1,2..,n-1\} de n\'umeros enteros y sea $\lambda$ un conjunto de enteros, $\lambda$=$\{\lambda_{1}, \lambda_{2},...,\lambda_{k}\}$, que satisface las siguientes propiedades:

\begin{itemize}
	\item $\lambda_{1}+\lambda_{2}+...+\lambda_{k}$=n.
	\item $\lambda_{i}\mid n$ $\forall$ i $\epsilon$ \{1,2,...,k\}
\end{itemize}

Definimos la transformaci\'on discreta de Baker $T_{N,\lambda}$ : N $\times$ N $\longmapsto$ N $\times$ N de la siguiente manera:

\begin{equation}
  T_{N,\lambda}(x,y) = [q_{i}(x-\sigma_{i}) + y \ mod \ q_{i} , \frac{1}{q_{i}}(y - y \ mod \ q_{s} ) + \sigma_{i}
\end{equation}

De donde $\sigma_{1}$ := 0 y $\sigma_{i}$ := $\lambda_{1}+...+\lambda_{i-1}$ para $2 \leq i < k$, $q_{i}$ := $\frac{n}{\lambda_{i}}$ y (x,y) $\epsilon$ [$\sigma_{i},\sigma_{i}+\lambda_{i})\times N $.

\end{defi}

Sin embargo al utilizar la transformaci\'on (1) en nuestro sistema no genero los resultados esperados; por ello siguiendo el esquema planteado por Jiri \cite{Jiri} de la forma en la cual se debe proceder para trabajar el mapa de Baker sobre una matriz se ha planteado la siguiente definici\'on alterna:

\begin{defi} Transformaci\'on discreta de Baker II :

Sea N en conjunto N=\{0,1,2..,n-1\} de n\'umeros enteros y sea $\lambda$ un conjunto de enteros, $\lambda$=$\{\lambda_{1}, \lambda_{2},...,\lambda_{k}\}$, que satisface las siguientes propiedades:

\begin{itemize}
	\item $\lambda_{1}+\lambda_{2}+...+\lambda_{k}$=n.
	\item $\lambda_{i}\mid n$ $\forall$ i $\epsilon$ \{1,2,...,k\}
\end{itemize}

Definimos la transformaci\'on discreta de Baker $B_{N,\lambda}$ : N $\times$ N $\longmapsto$ N $\times$ N de la siguiente manera:

\begin{equation}
  B_{N,\lambda}(x,y) = ( n - \sigma_{i} - \lfloor\frac{n-(x+1)}{q_{i}}\rfloor -1 , \frac{y-\sigma_{i}}{q_{i}} + [n-(x+1)] \ mod \ q_{i} ) 
\end{equation}

De donde $\sigma_{1}$ := 0 y $\sigma_{i}$ := $\lambda_{1}+...+\lambda_{i-1}$ para $1 \leq i < k$, $q_{i}$ := $\frac{n}{\lambda_{i}}$ y (x,y) $\epsilon$ $ N \times [\sigma_{i},\sigma_{i}+\lambda_{i})$.

\end{defi}

Seg\'un Jiri\cite{Jiri} la transformaci\'on $B_{N,\lambda}$ permuta la matriz como se sigue. Dada una matriz cuadrada de dimensi\'on \emph{N} se divide en $\mid\lambda\mid$ matrices de tama\~no $N\times\lambda_{i}$, cada una de las cuales a su vez es dividida en $\lambda_{i}$ matrices de dimensi\'on $\frac{N}{\lambda_{i}}\times\lambda_{i}$ con \emph{N} elementos, llamaremos a estas \'ultimas cajas, la propuesta de Jiri\cite{Jiri} es dada una caja se debe proceder de forma forma an\'aloga al mapa de Baker columna a columna. En ejemplo para una matriz de $8\times8$ es el siguiente. Supongase el conjunto $\lambda$=\{1,2,4,1\}.

\IncludeImage{./images/a}{130}{Configuraci\'on para $\Psi$}{a}

Una vez aplicamos la transformaci\'on $B_{8,\lambda}$ obtenemos.

\IncludeImage{./images/b}{130}{$B_{8,\lambda}(\Psi)$}{b}

Un aspecto importante para el trabajo es la dimensi\'on de la matriz de cifrado, llevandonos a la siguiente insignia.

\begin{insig}
 El sistema tendr\'a un excelente desempeño para N tales que el n\'umero de divisores sea m\'aximo.
\end{insig}

Para validar esta insignia se realizaron pruebas con matrices cuya dimensi\'on era un n\'umero primo dando como resultado que tan solo se generaban dos matrices para cualquier cantidad arbitraria de mensajes.

A este punto contamos con las herramientas necesarias para hablar del proceso que se sigue para generar la matriz de cifrado, para ello nos apoyamos de la noci\'on de \emph{funci\'on iterada}, que consiste en evaluar k-veces la funci\'on sobre si misma $f^{k}=f\circ ... \circ f $; un ejemplo, para \emph(k=3) $f^{3}=f(f(f(x)))$. La matriz de cifrado es generada en los siguientes pasos.

\begin{itemize}
 \item Por medio del mensaje de entrada se genera el valor \emph{u}.
 \item Se evalua la transformaci\'on de Baker iterando sobre ella \emph{u} veces, $B^{u}_{N,\lambda}(\Psi)$.
\end{itemize}

De donde el valor \emph{u} es obtenido de la siguiente manera.

$$u :=  (\alpha_{1}*(1 \ mod \ q)+\alpha_{2}*(2 \ mod \ q)+...+\alpha_{j}(n \ mod \ q)) \ mod \ j $$

Donde $q\approx 18*N$.

Para medir la sensibilidad del sistema frente a pequeños cambios, fueron ejecutadas 1000 pruebas con distintas permutaciones sobre un mensaje \emph{x} obteniendo que se generaba la misma matriz de cifrado con una probabilidad del 0.05. El mensaje usado en las pruebas fue el siguiente.

\begin{center}
``Los indigenas, que llegaron luciendo sus pinturas, plumas, arcos y flechas tradicionales, descendieron pacificamente del techo del Congreso poco despues, recorrieron la gran avenida donde se encuentran los ministerios y luego se sumaron a varios cientos de manifestantes anti Copa y del movimiento de los Sin Techo que marchaban hacia el estadio."
\end{center}



  Para la funci\'on de cifrado $f_{encryp}$ se requieren tres valores $\alpha$, $\phi$ y $\rho$, los cuales son el caracter a cifrar, el contador de iteraciones y el n\'umero de iteraciones que se realizar\'an al caracter $\alpha$ 
respectivamente.

El algoritmo de cifrado va a estar determinado por la posici\'on del caracter $\alpha$ en la matriz de cifrado, cada nueva posici\'on $x_{1}$ estaba dada por la ecuaci\'on basada en el fractal de Collatz \cite{Labelle} siguiente:

\[ x_{1} = 
\begin{cases} 
	\frac{x}{2} & \text {si } x \text { es par} \\ 
	3x+1 & \text {si} x \text {es impar} 
\end{cases} 
\] 

Sea $x_{1}$ la coordenada \emph{x} o \emph{y} de $\alpha$ en la matriz de cifrado.

El valor de $\rho$ es aleatorio, elegido en el rango de [0,81), dicho valor representa la cantidad de iteraciones que se har\'an al caracter $\alpha$, el cu\'al ser\'a necesario para la funci\'on de decifrado.
Despu\'es de cada iteraci\'on el valor de $\phi$ (siempre empezar\'a en 0) aumentar\'a en 1, el algoritmo se dentendr\'a cuando $\phi$ y $\rho$ sean iguales.
Se podr\'ia decir que un valor puede salirse de la dimensi\'on de nuestra matriz (y lo hace) por lo que realizamos una modificaci\'on: 

\[ x_{1} = 
\begin{cases} 
	\frac{x}{2} \text { mod  9} & \text {si } x \text { es par} \\ 
	(3x+1) \text { mod 9} & \text {si  x} \text { es impar} 
\end{cases} 
\] 
  

El m\'odulo en base 9 nos permite que cada coordenada siempre est\'e dentro de nuestra matriz de cifrado.
Al hacer esto notamos que al aplicar la funci\'on $f_{encryp}$ algunos valores llegaban al mismo destino; 1, 7 y 8 = 4 , 2 y 3 = 1, por lo que decidimos hacer un casteo de los valores que causaban conflicto para poder obtener una funci\'on biyectiva (condici\'on inicial de nuestra funci\'on $f_{encryp}$).

  \section{Transmisi\'on del mensaje}

Al igual que la funci\'on de encriptaci\'on la tranmisici\'on final del mensaje representa uno de los puntos cr\'iticos para el SCF. Al abordar esta tarea es necesario preguntarnos e intentar centrar nuestra discusi\'on en las siguientes dos preguntas:

\begin{itemize}
	\item ?`Qu\'e informaci\'on transmitir?
	\item ?`C\'omo la informaci\'on tramsmitida puede afectar la seguridad y validez del SCF?
\end{itemize}

En la introducci\'on al trabajo hablamos de la limitaci\'on propuesta por Howell \cite{Howell} sobre el decifrado del mensaje; con el fin de poder tener el control y tener recursos para desarrollar esta labor, se ha decidido que el mensaje final contiene los siguientes elementos.

\begin{itemize}
	\item Matriz de cifrado ($\Psi$).
	\item Cadena cifrada de \emph{x} ($x^{c}$).
\end{itemize}

Previo a la discusi\'on sobre la importancia de estos dos componentes es necesario conocer a forma de la cadena cifrada de \emph{x}. $x^{c}$ presentar\'a la siguiente configuraci\'on.

\begin{center}
\begin{tabular}{| c | c | c | c | c |}
	\hline
	$\mu_{1}$ & $\varphi_{1}$ & $\varphi_{2}$ & $\mu_{2}$ & ... \\
	\hline
	$\mu_{100}$ & $\varphi_{100}$ & $\varphi_{101}$ & $\mu_{101}$ & ...\\
	\hline
\end{tabular}
\end{center}

Es decir se transmitiran intercalados la car\'acteres cifrados y sus respectivas llaves, adem\'as de esto las llaves estar\'an dadas como un car\'acter de la matriz ($\Psi$); esto no representa problema alguno ya que como se vio en la funci\'on de cifrado la llave tomara los valores [0,N*N).

De ahora en adelante nos referiremos a $\varphi_{i}$ como la llave asociada al car\'acter $\mu_{i}$. Una vez conocidos los elementos que componen nuestro mensaje final es necesario conocer la importancia de cada uno dentro del proceso de decifrado, recordemos que un buen m\'etodo de cifrado debe comportarse de la siguiente manera.

\IncludeImage{./images/d}{100}{Comportamiento del sistema}{d}

La funci\'on de $\Psi$ dentro del mensaje final es vital como se menciono en la secci\'on para una cadena \emph{x} la matriz que se genera puede ser distinta a la de otro mensaje con una probabilidad alta; alguien podria pensar en reconstruir $\Psi$ a partir de $x^{c}$ sin embargo esta tarea puede representar un imposible dado que las condiciones iniciales de \emph{x} no existen y como se explico en la secci\'on 4 es necesario conocer \emph{x} para generar la matriz que le corresponde. Todo lo anterior se resume en la siguiente insignia.

\begin{insig}
	Para poder conocer $x$ a partir de $x^{c}$ es necesario conocer $\Psi$.
\end{insig}

Este resultado es importante para evaluar la seguridad del SCF. Al intentar vulnerar un sistema lo primero en lo que podemos pensar es tratar con fuerza bruta procediendo de la siguiente manera, dada $x^{c}$ probar con todas las posibles combinaciones para la matriz $\Psi$, mas este ataque es ejecutado en un tiempo de orden $(h*b)!$, donde \emph{h,b} representan el n\'umero de filas y columnas de $\Psi$.

La importancia de $\mu$ es algo trivial ya que sin este car\'acter no hay mensaje. Por otra parte la importancia de la llave esta dado el comportamiento de la funci\'on de cifrado y las propiedades del fractal, si no se tuviera el valor de esta llave. No se podria aplicar con presici\'on la transformaci\'on inversa, se podria intentar con todos los valores desde el 0 hasta el 80 pero esto represente un ataque de fuerza bruta poco eficiente.

  \section{Algoritmo para el descifrado}
Para descifrar la cadena usamos un algortimo 'simple', c\'omo mencion\'abamos anteriormente. Son necesarios dos caracteres; $\mu$ y $\varphi$, que representan al caracter cifrado y al n\'umero de iteraciones respectivamente.

Empezaremos con un contador $cont = 1$, el cu\'al indicar\'a si se encuentra en una posici\'on par o impar, este hecho modifica la lectura de $\mu$, dada de la siguiente manera:

\begin{center}
$\mu$, $\varphi$ Si \emph{cont} es par 
\end{center}
\begin{center}
$\varphi$, $\mu$ Si \emph{cont} es impar
\end{center}

Se debe tener en cuenta que $cont$ se aumenta en una unidad cada dos caracteres. Ubicamos a $\varphi$ en nuestra matriz $\Psi$ de cifrado y desciframos su valor num\'erico por medio de la ecuaci\'on:

\begin{equation}
 pos = (x * n) + y
\end{equation}

Siendo $n$ la dimensi\'on de nuestra matriz, en este caso $n = 9$. Como ya conocemos el valor num\'erico de $\varphi$ procedemos a descifrar $\mu$ el n\'umero de iteraciones correspondientes, para esto usamos la ecuaci\'on:

\[ \mu_{decrypted} = 
\begin{cases} 
	0 	& \text { si } \mu = 0 \\
	2*\mu 	& \text { si }  0 < \mu < 4\\
	3 	& \text { si } \mu = 5 \\
	8 	& \text { si } \mu = 6 \\
	\frac{\mu+8}{3} & \text { si } \mu = 7 \\
	6 	& \text { si } \mu = 8 \\
\end{cases} 
\] 

*Se tienen en cuenta los tres casos especiales 5, 6 y 8 que fueron casteados.

Dicha operaci\'on se realiza $\varphi$ veces con cada par de coordenadas ($x$, $y$). Al final se proceder\'a con transmitir el mensaje obtenido el cu\'al debe ser igual al original.

  \section{Conclusiones}

Al finalizar el desarrollo del trabajo podemos concluir:

\begin{itemize}
	\item El desarrollo de un SCF el cual pueda ser utilizado en procesos cotidianos es a\'un complejo y requiere de bastante investigaci\'on sobre la matem\'atica que rigen los cuerpos ca\'oticos y los fractales.

\end{itemize}

Queremos agradecer de manera especial al profesor Andres Villaveces, por su constante apoyo al trabajo.

  %BIBLIOGRAFIA
  %\newpage
  \bibliographystyle{alpha}
  \bibliography{./mod/biblio}
  \nocite{*}
  
\end{document}
